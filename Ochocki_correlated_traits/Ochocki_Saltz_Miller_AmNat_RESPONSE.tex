% ======================================================= %
% Document: TEMPLATE FOR RESPONSES TO REVIEWERS
% Author: Andrea Ballatore
% Date: Jan 7, 2013
% Source: https://raw.githubusercontent.com/ucd-spatial/Datasets/master/tex_response_to_reviewers_template/responses_to_reviewers.tex
% Modified by Eduard Szöcs, 10.03.2015
% ======================================================= %
\documentclass[12pt]{article}

% packages
\usepackage{xr}
\externaldocument[ms-]{Ochocki_Saltz_Miller_AmNat_REVISION}

\usepackage{graphicx}
\usepackage{url}
\usepackage[usenames,dvipsnames]{xcolor}
\usepackage{color}
\definecolor{mygray}{gray}{0.6}
\usepackage[utf8]{inputenc}
\usepackage[onehalfspacing]{setspace}
\usepackage[
	round,	%(defaultage in the main file and \input ) for round parentheses;
	colon,	% (default) to separate multiple citations with colons;
	authoryear,% (default) for author-year citations;
	sort,		% orders multiple citations into the sequence in which they
]{natbib}
\usepackage[%disable
	]{todonotes}

\usepackage{anysize}
\marginsize{2.5cm}{2.5cm}{1.5cm}{2.5cm}

% macros
% add a counter
\newcounter{cN}
\setcounter{cN}{0}

\newcommand{\comment}[1]{
	\vspace{2em}
	\refstepcounter{cN} % incrment counter
	\noindent \hangindent=0em \textbf{\textcolor{Maroon}{\uline{Comment \thecN}:~}}\emph{``#1''}
	}

\newcommand{\response}[1]{
	\\[0.25em]
	\hangindent=2.3em \textbf{\textcolor{NavyBlue}{\uline{Response}:~}}#1
	}

\newcommand{\revise}[1]{{\color{Mahogany}{#1}}}

\usepackage[normalem]{ulem}
\definecolor{darkred}{rgb}{1,.6,.6}
\DeclareRobustCommand\problemline{\bgroup\markoverwith{\textcolor{darkred}{\rule[-0.9ex]{4pt}{3pt}}}\ULon}
\DeclareRobustCommand{\problem}[1]{\problemline{#1}} % soul
\setcounter{secnumdepth}{-1}

\begin{document}
% ======================================================= %
\title{MS58835 --- Response to reviewers}

\maketitle
% ======================================================= %
\noindent To the editorial board,\\

Thank you for the opportunity to submit a major revision of our manuscript for your consideration. In this document, we reproduce comments from the editors and reviewers and provide our point-by-point responses. As we explain in more detail below, our responses and corresponding revisions focus on the following major concerns that emerged from our previous review.
\begin{enumerate}
\item{Reviewer 1 found some of our results unsurprising and questioned the novelty of our work relative to previous papers in this field. We now more clearly emphasize the novel results of our paper, highlighting these in a new sections of the Introduction and Discussion, as suggested by Dr. Winn, and we clarify the relationship between our work and related, foundational papers.}
\item{Following suggestions from Dr.'s Winn and Duckworth, we discuss possible mechanisms underlying the negative trait correlations that we documented experimentally. We also provide greater context for interpreting dispersal patterns in our laboratory environment, and we have created a new figure to more transparently represent our quantitative-genetic estimates and the raw observations on which they are based.}
\item{We provide new discussion of the limitations of our quantitative genetics experiments, specifically the inability to estimate maternal effects separately from additive genetic effects. In response to questions from Dr. Duckworth and Reviewer 2, we conducted new simulation analses to identify constraints on our abiity to estimate maternal effects.}
\end{enumerate}

All of our revisions are denoted in the manuscript with \revise{Mahogany font}.

\vspace{2em}
\hfill On behalf of all coauthors,

\hfill Tom Miller
\newpage



% ======================================================= %
\section{Response to Dr. Winn}
\vspace{-2em}

\comment{I was interested to read about your work using a model of range expansion under constraints due to covariance between dispersal and life history to interpret data from a lab empirical system and to explore more broadly how different patterns of covariance would be expected to alter the speed and trajectory of invasion.
The goals of your efforts are clearly important and relevant for our readers, and I found the paper to offer a well written narrative describing a powerful integration of modeling and empirical work.
However, I share the perspective from the reviews that your conclusions regarding how covariance would affect invasion do not seem entirely novel or particularly surprising.
I do think that providing confirmation of previous predictions using a different approach is a valuable contribution, and I also think your work goes beyond previous modeling efforts to show how to use these insights to help make sense of empirical observations. }
\response{We appreciate the opportunity to clarify the conceptual novelty and significance of our work relative to foundations in the literature, and we regret not having done this more effectively in our first submission.
While elements of our results do confirm familiar predictions (e.g., negative genetic correlations act as an evolutionary constraint), the core message of our paper is a novel one: trait correlations can alter expectations for the speed and variability of range expansion.
There is abundant evidence for demography-dispersal trade-offs in the range expansion literature, as emphasized by Reviewer 1, but no previous study has explored how such trade-offs affect ecological outcomes.
As an example of why this matters, our results are the first to show that, under some conditions, negative genetic correlations between dispersal and fertility can cause an evolutionary slow-down of range expansion, a new result that runs counter to most eco-evolutionary theory.
Furthermore, no previous studies have considered both positive and negative correlations spanning a single axis of covariation, or decomposed the contributions of genetic versus environmentally based covariance.
In our revision, we more clearly address these and other sources of novelty and significance (Intro paragraph beginning l.\ref{ms-mod2} and Disucssion paragraph beginning l.\ref{ms-mod3}).}

\comment{Thus, I agree with Dr. Duckworth that the paper could make a much stronger case if you connected your work more explicitly to its foundation in related published (and nearly published) work and also developed its potential to help interpret empirical patterns by further exploring the mechanisms that could drive your results for the bean beetle system.
In addition to her suggestion to expand your consideration of the particulars of your lab system, there is perhaps scope to make broader predictions about how past selection on either life history or dispersal propensity in any population would affect its invasion potential by shaping the pattern and strength of covariation it expresses.
Such a broader perspective might also incorporate the possibility for super-invaders, as suggested by Reviewer one.
From a purely editorial perspective, I would also suggest that a more compact text that emphasizes the novel insight your work offers over what has not yet been done could make the paper more compelling.
\\
}
\response{Thank you for the interesting ideas.
We have added a Discussion section addressing how historical factors can influence the expression of genetic covaration (paragraph beginning l.\ref{ms-mod5}).
We also now discuss implications for the evolution of `super-invaders' (paragraph beginning l.\ref{ms-mod4}).
Finally, we have taken the suggestion to compactly summarize the novelty and significance of our work in a new paragraph at the beginning of the Discussion (l.\ref{ms-mod3}).
}

\comment{In her comments to me, Dr. Duckworth summarizes the main concerns about the paper and along with the reviewers, offers other excellent suggestions for clarifying specific points in the paper.
Given the potential we see for the paper to make a contribution that would be widely appreciated by our readers, I look forward to seeing a revised manuscript that resolves all the issues raised in the full set of comments below.}
\response{Thank you for taking the time to provide thoughtful feedback on our manuscript.}

\section{Response to Dr. Duckworth}
\vspace{-2em}

\comment{We have now received two reviews and the reviewers were split in their assessment. Reviewer 2 was enthusiastic about the novelty of the study and the results; but Reviewer 1 was less so.
In particular, Reviewer 1 brings up several recent studies that have also done simulation studies on dispersal and invasion dynamics.
While the current study cites many of these recent papers, it could do a better job of explaining the differences in approaches with the current paper and how that may affect interpretation of results.}
\response{We have added new content that clarifies the novelty and significance of our study with respect to foundational papers in this area (l. \ref{ms-mod2},\ref{ms-mod3}).
As we explain in these new paragraphs, previous studies have incorporated trade-offs into eco-evolutionary range expansion models, but none have examined how trade-offs (or trait correlations more generally) affect expansion speed and variability, and none have decomposed the relative roles of genetic and environmental correlations.
%If our focus were strictly on trait evolution, then we agree that our finding that a negative genetic correlation constrained evolutionary responses during spread is neither new nor surprising.
%However, as we now more clearly explain, this is not the main point of our paper.
%Instead, we emphasize how trait correlations and their role in eco-evolutionary feedbacks can affect ecological outcomes of expansion speed and variability.
}

\comment{The current study should also evaluate the Fronhofer and Alternatt (2015) paper in particular and discuss how the current paper builds on and/or differs from the approach and conclusions there.}
\response{Thank you for bringing this to our attention. Upon re-reading Fronhofer and Altermatt (2015), we agree that we did not give this paper proper billing in our original submission.
In fact, we even made the incorrect assertion that only two studies have incorporated trait correlations into eco-evolutionary models of range expansion; Fronhofer and Altermatt (2015) is now included as a third (paragraph beginning l.\ref{ms-mod2}).
We also now revisit the findings of Fronhofer and Altermatt (2015) in our Discussion (l. \ref{ms-mod6}).
}

\comment{What I missed and I think is also implicit in some of Reviewer 2’s comments was a deeper discussion of the potential mechanisms underlying the results of both the simulation and the QG analysis.
Dispersal is comprised of a multitude of traits with unfolding over distinct stages (propensity to disperse, actual costs of moving, costs of settling, etc).
The authors mention this very briefly in the discussion (line 492-6) but admit they have nothing to say on this.
What are the possible costs of dispersal in this system?
In the uniform lab environment is the variation we observe among individuals due simply to their intrinsic propensity to move farther?
Or do interactions with other individuals in these lab systems impact the costs of dispersal and female fecundity?
It would be useful to have more details on what is known about the dispersal environment of the lab populations and how this may or may not apply to natural populations.}
\response{We understand these comments from the AE and reviewers, and our revision addresses this in several ways. 
First, we have added a new figure to more transparently communicate the traits that we are modeling.
Our previous submission showed only the parameter estimates from the QG analysis; based on comments raised in our previous review (including the AE's next comment), we think that showing the underlying data may help put these parameter estimates into context.
Second, we have added information about what we know about density-dependent movement in our system and we discuss how genetic variation in dispersal may reflect either innate movement tendencies or responses to conspecifif density (paragraph beginning l.\ref{ms-mod10}).
Finally, while it remains the case that we do not have a clear mechanistic understanding of costs of dispersal in our system, we now discuss some hypotheses for this (l.\ref{ms-mod11}) and for how our lab results might translate to natural populations (l.\ref{ms-mod12}).
}

\comment{Finally, I agree with Reviewer 1 that it is strange that the maternal effect model did not converge.
Typically, this is because a model is over-specified, but that doesn’t seem to be the case here.
It makes me wonder if there is something strange in how the experimental set-up creates variance among sibs in dispersal.
Given that all groups in the trials are full sibs, if they all use a simple rule of dispersing until they no longer encounter a sibling or conspecific, this could lead to identical distributions of dispersal distances among sib groups, leading to high within relative to among-brood variance and also calling into question the biological relevance of the experimental set-up.
What are the ranges of dispersal in the system? The variation in both dispersal and fecundity looks extremely low overall – could this be the issue?
More details on the how these traits vary in the lab environment is needed to assess the QG analysis.}
\response{As mentioned in our previous response, we have added a new figure that shows the empirical observations to help put our QG results into context.
Regarding the AE's concerns about variance being greater within versus among groups of full sibs, this precisely what our heritability estimates are meant capture.
While it is fair to question the biological relevance of our dispersal set-up in this lab system, the results of this study and our previous work (Ochocki and Miller 2017) clearly indicate that, in this context, there is significant among-brood variance in dispersal. \\
\\
Like the AE and Reviewer 2, we were surprised and disappointed that we could not estimate maternal effects from our data.
In responses to these comments raised in our review, we have dug into this issue with new simulation work (not added to the manuscript because it is tangential to the scope, though we are open to including this).
\revise{Summarise findings.}
}

\section{Response to Reviewer 1}
\vspace{-2em}

\comment{In the simulation model presented here, evolution of traits is – as I understand it - limited by the given additive genetic variation, i.e. evolution is restricted to the effect of selection on the initial standing genetic variation.
This allows to model the evolution of reproductive rate without the assumption of any trade-offs (which generally prevent the evolution of super-fertile and super dispersive organisms) but restricts the validity of simulation outcomes.
This type of model is in line with the model used by Phillips (2015) who points out that this type of model necessarily focuses on small population sizes and short-term evolutionary dynamics, since it assumes that genetic variances are eroded by selection, but are not subject to the additional force of mutation.
I completely agree that this approach may be adequate for fort-term processes at the front of an expanding population.
However, this restriction should be discussed more extensively as it makes it more or less impossible to compare the simulation results presented here with those of more general (and rather influential) evolutionary models (e.g. Travis and Dytham 2002, Burton et al. 2010, Fronhofer and Altermatt 2015, Shaw and Kokko 2015).}
\response{We agree that our assumption that selective processes act exclusively on standing genetic variation is only appropriate for short time-scales and creates some difficulty in comparing our results to others models that include novel mutations. We now include this in our Discussion (l. \ref{ms-mod7}).}

\comment{The authors mention in the discussion section (l. 377 ff) that ``...different bodies of literature emphasize connections between demography and dispersal traits through a myriad of mechanisms, ranging from trade-offs and costs of dispersal (typically corresponding to negative correlations) to dispersal 'syndromes' that package several life history and movement traits into a multivariate phenotype (typically corresponding to positive correlations).''
In the context of these ``bodies of literature'' I found the general results of the present manuscript not particularly innovative or counter-intuitive.
It is generally assumed (and not surprising) that, all cost being ignored, selection will favor high fertility as well as high dispersal rates at the front of populations expanding into empty suitable habitat.
However, if increased dispersal comes at the cost of reduced fertility (negative correlation), selection on increased dispersal will be reduced and – depending on this cost (the strength of this correlation) – selection may even favor increased fertility and reduced dispersal instead.}
\response{We agree that the trait evolution results are not particularly surprising but we emphasize that these are only a subset of our results.
As we now clarify, the more important contribution of our work is how trait correlations affect higher-level expansion outcomes in terms of speed and variability (l. \ref{ms-mod3}).
While perhaps not surprising, we include the trait evolution results because we feel they are a critical part of the story.
Even if the idea that trade-offs constrain trait evolution is familiar, there is no consensus from theory on whether dispersal or reproductive rate should take evolutionary precedence on expanding front, a point reinforced by Reviewer 2.
In this sense, our trait evolution results (it was always dispersal that increased at the expense of fertility) provide new insights, which we now highlight (new paragraph beginning l. \ref{ms-mod17}).}

\comment{For a more general discussion of evolutionary forces working at the front of an expanding population the results of the simulations presented here are - to my opinion - not really helpful.
On the contrary: in comparison to the models of e.g. Burton et al. (2010) or Fronhofer and Andermatt (2015) I missed the consideration of trade-offs that limit the evolution of dispersal ability and fertility and prevent the evolution of ``super-organisms''. Hanski et al. 2006, Chuang et al. (2016), Phillips et al (2016), and Perkins et al (2016) all point out the importance of trade offs and Chuang et al. (2016) states that ``Accounting for the effects of edge phenotypes and related trade-offs could be critical for predicting the spread of invasive species and population responses to climate change.''}
\response{We understand the reviewer's point here but we would like to emphasize important distinctions between our study and the papers that this reviewer cites.
The reviewer notes that these studies all point out the importance of trade-offs.
This is true, but only in the context of trait evolution, which is one side of the eco-evolutionary coin; none of these studies (or any others to our knowledge) test how ecological outcomes of expansion speed and variability respond to variation in the sign and magnitude of trait correlations, including those that are genetically versus environmentally based; we make this critical source of novelty much clearer in our revision and we contrast our work with the studies that the reviewer suggests (paragraph beginning l. \ref{ms-mod3}).
Burton et al. (2010) built a simulation model that assumed a zero-sum tripartite trade-off among reproduction, dispersal, and response to density; they show that this trade-off favored increased dispersal and reproduction and decreased competitive ability at the leading edge.
Because their analysis assumes a perfect negative correlation, they cannot evaluate how variation in the sign, magnitude, and basis (genetic or environmental) of trait correlations affect the speed of expansion.
The same is true of Fronhofer and Altermatt (2015), whose experiment showed that evolved increases in dispersal at the leading edge are associated with decreases in foraging, and whose eco-evolutionary model assumes a perfect negative correlation between these traits.
Perkins et al. (2016) do consider expansions with and without dispersal-life history trade-offs but, again, their focus is on trait evolution (particularly the attenuation of local dispersal ability post-colonization) and not on the ecological outcomes of expansion speed and variability.
Hanski et al. (2006) and Chuang and Peterson (2016) are empirically oriented studies that document the occurrence of dispersal trade-offs; we cite these studies to motivate our work on the consequences of trait correlations for eco-evolutionary outcomes, but those studies themselves do not address such outcomes.
We could not find a relevant 2016 Phillips paper but perhaps the reviewer intended Hudson et al. 2016, which documents dispersal-reproduction trade-offs in cane toads; we have added this citation to the paper (l. \ref{ms-mod8}).
Overall, we are confident that our work makes a novel contribution with respect to the previous literature and we hope that our revisions on this issue convince the reviewer, and readers more generally.
\\
\\
The idea of trade-offs limiting the evolution of ``super-organisms'' is very interesting and we have added this idea to the Discussion section (paragraph beginning l. \ref{ms-mod4}).}


\comment{l. 35 Concerning the statement ``Low-density conditions of the vanguard can also result in natural selection for increased reproductive rates ('r-selection').'' and similarly line 360 it might be interesting to discuss the results of Fronhofer and Altermatt 2015 (Eco-evolutionary feed-backs during experimental range expansions. Nature Communications 6:6844).
These authors state ``Interestingly, the experimentally and theoretically observed distribution of population densities across a species' range seems counter-intuitive, as population densities increase from low densities in range cores to high densities at range margins.''}
\response{We have updated this sentence to make it clear that the vanguard of an expansion need not geneally be at low density (l. \ref{ms-mod9}). 
Based on our understanding of the Fronhofer and Altermatt results, the edge of their expansions are low-density, but it is right behind the edge that they observe increased density (their Fig 3a and 4d). 
If so, then the low-density dynamics we describe may still apply to the Fronhofer and Altermatt study / model.}

\comment{l. 38 better: ``is expected to increased''}
\response{Fixed.}

\comment{l 48. better: ``Then it is impossible''}
\response{Fixed}

\comment{1 140. better: ``make this species a popular model''}
\response{Fixed.}

\comment{line following eq.3: Setting $Nt=1$ and dividing both side by B gives}
\response{Done.}

\comment{l. 325 better: ``increases in dispersal ability''}
\response{Fixed.}

\comment{l. 326 better: ``decreases in fertility''}
\response{Fixed.}

\comment{l. 399 better:``increases in mean''}
\response{Fixed.}
\\
\\
\noindent Thank you to this reviewer for taking the time to carefully review our paper.
Their comments and suggestions have improved the paper.

\section{Response to Reviewer 2}
\vspace{-2em}

\comment{This is a really excellent study.
It makes the point that genetic covariance between demography and dispersal can do interesting things to invasion dynamics, and then provides a pioneering and laudably thorough exploration of this idea using a neat combination of empirical and simulation work.
There are some very recent theoretical developments that are worth the authors looking at, but otherwise, there is very little to improve here.
\\
\\
\textbf{Introduction}
The introduction is a tour de force. Does a great job of introducing the topic, and identifies very nicely where this particular piece of work is aimed.}
\response{We appreciate the positive feedback and the suggestions to include recent theoretical developments, which we have done.}

\comment{
\textbf{Methods}
\textbf{QG Analysis}
The quant genetics design is a classic full/half-sib design. Well executed, and with sufficient sample sizes (150 unique full-sib families, each nested within one of 50 half-sib families; 16 individuals, balanced for sex within each full sib family).
\\
Clarification: “After 28 days of development. . . Within 48 hours of eclosion. . . ”. Slightly confusing. If I understand, dispersal was done 48 hours after eclosion, and fertility taken as number of offspring generated in the following 28 days.}
\response{This interpretation is correct. We have revised this section for clarity (l. \ref{ms-mod13}).}

\comment{Surprising that the maternal effects model would not converge. You should have sufficient data.
Still, if it won’t converge...}
\response{\revise{Waiting on sim results.}}

\comment{Eqn 4: I think you should drop the hat from $N_{t+1}$ because it isn’t an estimator here, it is a parameter.
Though conventions vary.}
\response{Done.}

\comment{``and is distributed [on the latent scale, log in this case] according to a multivariate normal distribution centered on the averag''.
Yes? If so, it might be simplest to just state earlier that your ``latent'' scale is a log scale for all traits.}
\response{It is true that, because both traits are Poisson distributed, they both use a log link to translate the observations to the latent scale.
However, we have opted against implmenting this suggestion because there is more to the latent scale issue than just the log link.
The latent scale is unobservable, but we can draw inferences about it via stochastic realizations.
We don't want to imply that the latent scale is simply the log of the dispersal distances or offspring counts.
The particular statistical transformation is the less important piece of it.}

\comment{rstan in text citation seems to have formatted incorrectly.}
\response{\revise{I can't figure this out!}}

\comment{Great to see all the code up and available on github!}
\response{We are glad that some readers find this useful.}

\comment{\textbf{Simulations}
\\
There is not a great deal of detail here, though the details in the appendix and code repository are sufficient.}
\response{As long as the reviewer thinks the current layout is sufficient then we have made no changes, as the manuscript is already quite a bit longer due to additions in the Discussion}

\comment{\textbf{Results and Discussion}
The results are really nice. First, we have some estimates for genetic covariation between dispersal and reproduction.
Second, we have simulation results that give qualitatively similar results to those in the much larger invasion experiment reported in the Nature paper.
Finally, we have the really cool result that invasions accelerate even with strong negative covariation.
In this regard, the results are confirmation of a very recent result from an analytic model of Phillips and Perkins, which explain why this is expected (latest pre-print is at https://www.biorxiv.org/content/early/2018/12/11/210088 and recently accepted pending minor changes in Theoretical Ecology).}
\response{Thank you for bringing this recent reference to our attention. 
We have added this important citation where the reviewer suggest (l. \ref{ms-mod14}) and elsewhere.}

\comment{The results also examine variance in spread rate (presenting the results in a beautifully elegant manner), and this too is the subject of a very recent analytic paper (by Peischl and Gilbert: https://www.biorxiv.org/content/early/2018/11/30/483883 ).
Most of the result here line up quite nicely with these pieces of theory, I believe (though I haven’t been through Peischl’s stuff in detail yet).
The clear exceptions are a) the possibility of deccelerating spread, and b) muted fertility response compared with dispersal; really interesting results.}
\response{Thanks again for bringing a relevant new paper to our attention.
This paper actually does not include results on variance in spread rate.
They focus on expansion load, which can be a driver of variance in speed, but the authors do not make this connection.
However, we now cite this paper where we discuss mechanisms of evolutionary slow-down (l. \ref{ms-mod15}) and where we acknowledge a potentially important role of mutation (1. \ref{ms-mod7}.)}

\comment{Line 434-41: Nice. It might be worth couching this argument with reference to Fisher’s result that invasion speed is $2\sqrt{rD}$. Your argument is cogent, but might be more clearly put against this backdrop.
Again, I think those two preprints (above) might help here.}
\response{Thank you, we now refer to the classice diffusion result and the more recent papers to help contextualize these results (l. \ref{ms-mod16}).
\\
\\
As an aside to the editors, while we describe the classic Fisher result in our paper we chose not to cite it because this study was published in the Annals of Eugenics and we do not want this reference in our paper.
Our hope is that this result is now so canonical and so closely associated with Fisher that the explicit reference is unnecessary.
}

\comment{455-460: The dominant response of dispersal is an intriguing result and I agree that it lines up well with the growing body of empirical work.
It is not predicted by the analytic model of Phillips and Perkins, but the key difference may be density dependence (which you have and they don’t).
Might be worth mentioning this possibility (with density dependence, the force of r-selection may diminish faster at a location than the force of spatial sorting: I imagine that at very low r, high K, we might see r-selection hold its own against spatial sorting. Maybe.)}
\response{This is an interesting idea that we have incorporated into a new Discussion paragraph (beginning l. \ref{ms-mod17}).}

\comment{Overall, a really thoughtful discussion and an excellent piece of science.}
\response{Thank you for the constructive feedback and encouragement, and for taking the time to carefully review our paper.}

%% --------------------------------
%\newpage
%\bibliography{refs}
%\bibliographystyle{spbasic}


% ======================================================= %
\end{document}
% ======================================================= %
